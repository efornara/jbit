\documentclass[a4paper,twocolumn,11pt]{article}
\usepackage[latin1]{inputenc}
\usepackage[top=3cm,bottom=3cm,left=1.5cm,right=1.5cm]{geometry}
\usepackage{fancyhdr}

\begin{document}

\pagestyle{fancy}
\fancyhead{}
\fancyfoot{}
\lhead{\small JBit-IO (R)}
\rhead{\small Registers, Constants, Requests}
\lfoot{\small Copyright \copyright\ 2007-2016 Emanuele Fornara}
\cfoot{\small JBit}
\rfoot{\small \tt http://jbit.sourceforge.net/}

\section*{\centerline{JBit IO (R)}}

\subsection*{Registers}

\begin{verbatim}
REQPUT      1        CONVIDEO   40(40)
REQEND      2        CONROW0    40(10)
REQRES      3        CONROW1    50(10)
REQPTRLO    4        CONROW2    60(10)
REQPTRHI    5        CONROW3    70(10)
ENABLE     16        LID        80
FRMFPS     17        LCTL       81
FRMDRAW    18        LX         82
GKEY0      19        LY         83
GKEY1      20        SFRAME     84
RANDOM     23        STRANSFM   85
KEYBUF     24(8)     SCWITH     86
CONCOLS    32        TCOLLO     87
CONROWS    33        TCOLHI     88
CONCX      34        TROWLO     89
CONCY      35        TROWHI     90
CONCCHR    36        TCELL      91
CONCFG     37        REQDAT     96(32)
CONCBG     38
\end{verbatim}

\subsection*{Constants}

\noindent {\bf ENABLE (Bitmask)}
\begin{verbatim}
BGCOL       1        BGIMG       2
CONSOLE     4        LAYERS      8
\end{verbatim}

\noindent {\bf Standard Palette}
\begin{verbatim}
BLACK       0        ORANGE      8
WHITE       1        BROWN       9
RED         2        LIGHTRED   10
CYAN        3        GRAY1      11
PURPLE      4        GRAY2      12
GREEN       5        LIGHTGREEN 13
BLUE        6        LIGHTBLUE  14
YELLOW      7        GRAY3      15
\end{verbatim}

\noindent {\bf Line Art (Bitmask)}
\begin{verbatim}
TOP         1        BOTTOM      8
LEFT        2        RIGHT       4
CDOT      128
\end{verbatim}

\noindent {\bf Datatypes}
\begin{verbatim}
U8          1        I8          2
U16         3        I16         4
\end{verbatim}

\vskip 1em

\noindent {\bf LCTL - Layer Control (Bitmask)}
\begin{verbatim}
SHIFTX0     1        SHIFTX1     2
SHIFTY0     4        SHIFTY1     8
PXLCOLL    16        ENABLE    128
\end{verbatim}

\noindent {\bf STRANSFM - Sprite Transformation}
\begin{verbatim}
NONE        0        ROT90       5
ROT180      3        ROT270      6
MIRROR      2        MROT90      7
MROT180     1        MROT270     4
\end{verbatim}

\noindent {\bf GKEY0 - Game Keys 0 (Bitmask)}
\begin{verbatim}
UP          2        LEFT        4
RIGHT      32        DOWN       64
\end{verbatim}

\noindent {\bf GKEY1 - Game Keys 1 (Bitmask)}
\begin{verbatim}
A           2        B           4
C           8        D          16
FIRE        1
\end{verbatim}

\subsection*{Requests}

On the other side of this sheet you can find the syntax of the requests
(and the corresponding results, when applicable).
See the {\em bgcol1} and {\em bgcol2} demos for examples of how
to send a request to the IO chip.
After the request has been sent, REQRES
contains 0 on success and 255 on failure (usually not tested).
Results are available starting from REQDAT.
Streaming requests (identified by a {\tt >}) are not bounded.
The other requests are bounded (255 bytes).

Optional values are delimited by [ and ]. ( and ) are used for grouping.
* means repeat 0 or more times. + means repeat at least once. \# means
repeat with constraints. $|$ means choice (priority is low).
Datatype is U8 unless stated otherwise (by a tag preceded by :).
For datatypes larger than 8 bits, the least significat byte comes first.
Enumerated values are identified by C (for choice).
Bitmasks are identified by O (for OR).
Strings can be delimited by 0 (S0) or by the end of the request (S).
When an argument has datatype T, the actual datatype is
chosen by the user with DType (see constants above).

For the semantic of the requests, take a look at the demos
or simply experiment using names as hints.
Notes for the IPNGGEN request:
using INDEXED\_COLOR causes a PLTE chunk to be generated
(a pallette must be provided) and
setting IDX0TRANSP causes a tRNS chunk to be generated.
For more information see the PNG specification.

\clearpage

\begin{flushleft}
{\bf System}
\vskip 2pt
{\tt NOREQ(0)>}: Dummy* \\
\vskip 4pt
{\tt TIME(2)}: [RefTime=ABS [Fract=1000]] \\
RefTime(C): {\tt ABS(1)}, {\tt RESET(2)} \\
Fract(C): {\tt 1(1)}, {\tt 10(2)}, {\tt 10(3)}, {\tt 1000(4)} \\
Result: Time:U64 \\
\vskip 4pt
{\tt LOADROM(6)}: Addr:U16 ResName:S0 [Offset:U16 Size:U16] \\
\vskip 4pt
{\tt RSFORMAT(8)}: 121 33 \\
\vskip 4pt
{\tt RLOAD(9)}: Addr:U16 Size:U16 RecName:S0 \\
\vskip 4pt
{\tt RSAVE(10)}: Addr:U16 Size:U16 RecName:S0 \\
\vskip 4pt
{\tt RDELETE(11)}: RecName:S0 \\
\vskip 6pt
{\bf Display and Imaging}
\vskip 2pt
{\tt DPYINFO(16)}: - \\
Result: Width:U16 Height:U16 ColorDepth AlphaDepth Flags \\
Flags(O): {\tt ISCOLOR(128)}, {\tt ISMIDP2(64)} \\
\vskip 4pt
{\tt SETBGCOL(17)}: PaletteEntry $|$ Red Green Blue \\
\vskip 4pt
{\tt SETPAL(18)>}: (Red Green Blue)* \\
\vskip 4pt
{\tt SETBGIMG(19)}: ImageId \\
\vskip 4pt
{\tt IDESTROY(20)}: ImageId \\
\vskip 4pt
{\tt IDIM(21)}: MaxImageId \\
\vskip 4pt
{\tt IINFO(22)}: ImageId \\
Result: Width:U16 Height:U16 Flags \\
Flags(O): {\tt ISMUTABLE(128)} \\
\vskip 4pt
{\tt ILOAD(23)}: ImageId ResName:S0 \\
\vskip 4pt
{\tt IDUMMY(24)}: ImageId Type=SIMPLE Width:U16 Height:U16 Bg Fg [Title:S] \\
{\tt IDUMMY(24)}: ImageId Type=SPRITE Width Height Frames Bg Fg [Title:S] \\
{\tt IDUMMY(24)}: ImageId Type=TILES Width Height Cols Rows Bg Fg (N Bg Fg)* \\
Type(C): {\tt SIMPLE(1)}, {\tt SPRITE(2)}, {\tt TILES(3)} \\
\vskip 4pt
{\tt IPNGGEN(25)>}: ImageId  Width:U16 Height:U16 Depth ColorType Flag
[MaxPaletteEntry (PaletteEntry $|$ Red Green Blue)\#] Data\# \\
ColorType(C): {\tt GRAYSCALE(0)}, {\tt TRUECOLOR(2)}, {\tt INDEXED\_COLOR(3)},
{\tt GRAYSCALE\_ALPHA(4)}, {\tt TRUECOLOR\_ALPHA(6)}\\
Flags(O): {\tt IDX0TRANSP(1)}, {\tt PALREF(2)}, {\tt ZOOM0(4)}, {\tt ZOOM1(8)} \\
\vskip 4pt
{\tt IEMPTY(26)}: ImageId Width:U16 Height:U16 \\
\vskip 4pt
{\tt IMKIMMUT(27)}: ImageId \\
\vskip 4pt
{\tt IRAWRGBA(28)>}: ImageId Width:U16 Height:U16 Flags
(Red Green Blue Alpha)\# \\
Flags(O): {\tt ALPHA(128)} \\
\vskip 6pt
{\bf Layers (Game API)}
\vskip 2pt
{\tt LMVIEW(32)}: TiledLayerId $|$ DType [X:T Y:T] Width:T Height:T \\
\vskip 4pt
{\tt LMPOS(33)}: DType OX:T OY:T \\
\vskip 4pt
{\tt LDESTROY(34)}: LayerId \\
\vskip 4pt
{\tt LDIM(35)}: MaxLayerId \\
\vskip 4pt
{\tt LTILED(36)}: TiledLayerId ImageId TWidth THeight NAnimTiles DType
Cols:T Rows:T \\
\vskip 4pt
{\tt LSPRITE(37)}: SpriteId ImageId [Width Height] \\
\vskip 4pt
{\tt LSETPOS(38)}: LayerId DType X:T Y:T \\
\vskip 4pt
{\tt LGETPOS(39)}: LayerId \\
Result: X:I32 Y:I32 \\
\vskip 4pt
{\tt LMOVE(40)}: LayerId DType DX:T DY:T \\
\vskip 4pt
{\tt LSETPRI(41)}: LayerId DType Priority:T \\
\vskip 4pt
{\tt LGETPRI(42)}: LayerId \\
Result: Priority:I32 \\
\vskip 4pt
{\tt LTLANIM(43)}: TiledLayerId AnimTile StaticTile\\
\vskip 4pt
{\tt LTLFILL(44)}: TiledLayerId Tile [DType Col:T Row:T NumCols:T NumRows:T] \\
\vskip 4pt
{\tt LTLPUT(45)>}: TiledLayerId Col:U16 Row:U16 NumCols:U16 Tile* \\
\vskip 4pt
{\tt LTLSCRLL(46)}: TiledLayerId ScrollType={\tt 0} DType Col:T Row:T NumCols:T NumRows:T
DX:T DY:T \\
\vskip 4pt
{\tt LSPCOPY(47)}: SpriteId TemplateSpriteId \\
\vskip 4pt
{\tt LSPAPOS(48)}: SpriteId DType AbsX:T AbsY:T \\
\vskip 4pt
{\tt LSPREFPX(49)}: SpriteId DType RefPixelX:T RefPixelY:T \\
\vskip 4pt
{\tt LSPCLRCT(50)}: SpriteId DType CollRctOX:T CollRctOY:T CollRctWidth:T CollRctHeight:T \\
\vskip 6pt
{\bf Effects}
\vskip 2pt
{\tt FXTONE(64)}: Duration Frequency Volume \\
\vskip 4pt
{\tt FXVIBRA(65)}: Duration \\
Result: Supported \\
\vskip 4pt
{\tt FXFLASH(66)}: Duration \\
Result: Supported \\
\vskip 4pt
\end{flushleft}

\end{document}
